\documentclass[10pt,twoside]{fernandes_supp} 
%Default opts:9pt,twocolumn,twoside
\graphicspath{{images_supp/}}

\newcommand{\KB}[1]{\noindent\color{blue}$\Longrightarrow$ #1\normalcolor}
\newcommand{\MF}[1]{\noindent\color{ForestGreen}#1\normalcolor}

\title{Supplemental Information:\\ Harnessing  Design Principles from Glass Sponges for Structurally Robust Lattice Structures}

\author[1]{Matheus C. Fernandes}
\author[2]{James C. Weaver}
\author[1,3,*]{Katia Bertoldi}

\affil[1]{John A. Paulson School of Engineering and Applied Sciences -- Harvard University, Cambridge, MA 02138}
\affil[2]{Wyss Institute -- Harvard University, Cambridge, MA 02138}
\affil[3]{Kavli Institute -- Harvard University, Cambridge, MA 02138}
\affil[*]{Corresponding author: \href{mailto:bertoldi@seas.harvard.edu}{bertoldi@seas.harvard.edu}}

\begin{document}
\maketitle
% \doublespace
% \linenumbers
\section{Structure of the Hexactinellid sponge \textit{Euplectella aspergillum}}\label{sec:params}
The periodic structures investigated in this study are inspired by the skeleton of the hexactinellid sponge \textit{Euplectella aspergillum (sp.)}, commonly known as the "Venus' flower basket". In this section we provide a detailed description of the sponge geometry and measured dimensions.

\Cref{Sponge} shows a photograph of the entire skeleton of the \textit{Euplectella sp.}, and its intricate, cylindrical cage-like structure (20 to 25cm long, 2 to 4 cm in diameter) \citep{aizenberg2005}. The surface of the cylinder consists of a regular square lattice composed of a series of cemented vertical and horizontal struts with circular cross-section. The cell spacing between horizontal and vertical struts was reported  to be $L\approx 2.5$mm \citep{weaver2007}, while their diameter was measured to be $D_{nd}\approx 0.25$mm \citep{weaver2007}. Besides the horizontal and vertical struts, there is an additional set of diagonal elements, intersecting in a manner that creates a series of alternating open and closed cells, reminiscent of a checkerboard pattern \citep{weaver2007}. Although these diagonal elements are not as ordered as the horizontal and vertical ones, it has been shown that they can be approximated with two diagonal struts that are offset from the nodes  (vertex joints between non-diagonal elements) and form octagonal openings (\cref{Sponge}(d)). To estimate the volume ratio between diagonal and non-diagonal elements, we took high resolution photographs of the sponge and performed image segmentation to segregate the projected area of the vertical/horizontal and diagonal spicules. Using this approach, the projected area ratio of non-diagonal to diagonal elements was found to be $A_{nd}/A_{d}\approx1.4$. Note that here, and in the following, the subscripts $d$ and $nd$ are used to indicate diagonal and non-diagonal (i.e. horizontal and vertical) elements, respectively. 

Finally, it should also be noted that the sponge is reinforced by external ridges that extend perpendicular to the surface of the cylinder and spiral the cage at an angle of $45^\text{o}$. However, in this paper we do not report the effects of these ridges on it's structural performance %and treat the periodic structure as a planar (no out of plane deformation) unit cell.
% \figll{SpongeCylinder}{0.35}{Close-up photograph of Hexactinellid sponge Euplectella aspergillum.}
\figll{Sponge}{0.9}{\textbf{Hexactinellid sponge \textit{Euplectella aspergillum.}} (a)-(b) Full-frame photo of sponge.  (c)  Close up microscope image of the sponge. (d) Comparison between the idealized model (green and blue lines) and the sponge structure. (e) Unit cell of the idealized model.}

\section{Designs Considered}
In this study, we consider four different lattice configurations using in two different elements cross-sections constrained to deform in an in-plane setting only. In an effort to provide a fair performance compassion between the different designs all of the respective lattice configurations the total amount of material (i.e. volume or mass) is conserved. Moreover,  we consider a fixed volume ratio between non-diagonal and diagonal elements, chosen to match the sponge geometry. In the subsequent sections, we describe in detail the unit cells for the four different designs (Designs A-D).

%In this study, we consider four different planar periodic lattice structures and compare their performance in terms of buckling resistance and stiffness. All structures are made of struts with circular cross section and are assumed to only deform in-plane.  In an effort to provide a fair performance comparison between all four designs, we scale the different diameters of each design to maintain a constant total volume for each unit cell. Moreover,  we consider a fixed volume ratio between non-diagonal and diagonal elements, chosen to match the sponge geometry. In the subsequent sections, we describe in detail the unit cells for the four different designs (Designs A-D).

\subsection{Polygonal Cross-section}
\subsubsection{Design A}
From $A_{nd}/A_d\approx 1.4$ we know that the volume is the same given that the in-plane dimension is constant, namely $V_{nd}/V_d\approx 1.4$ . Assuming that the in-plane thickness is given by $t$, we can derive the following volumes for the diagonal and non-diagonal, respectively:
\begin{equation}
	V_{A,nd}=8LD_{A,nd}t
\end{equation}
\begin{equation}
	V_{A,d}=8\sqrt{2}LD_{A,d}t
\end{equation}
Let us enter this into our biological observation, which yields:
\begin{equation}
	{V_{A,nd}\over V_{A,d}}\approx 1.4\approx\sqrt{2}={8LD_{A,nd}t\over8\sqrt{2}LD_{A,d}t}
\end{equation}
Therefore, we can simplify this relationship as
\begin{equation}
	D_{A,nd}=2D_{A,d}
\end{equation}

\subsubsection{Design B}
For this design the volume of non-diagonal elements remain the same, namely,
\begin{equation}
		V_{B,nd}=8LD_{B,nd}t.
\end{equation}
However, the volume for the diagonal elements will is different as a result the change in total length, namely,
\begin{equation}
		V_{B,d}=4\sqrt{2}LD_{A,d}t.
\end{equation}
In in effort to maintain a constant volume ration between the diagonally reinforced designs, we use the same volume ratio as per observation
\begin{equation}
		{V_{B,nd}\over V_{B,d}}\approx 1.4\approx\sqrt{2}={8LD_{B,nd}t\over4\sqrt{2}LD_{B,d}t}
\end{equation}
which simplifies to
\begin{equation}
	D_{B,nd}=D_{B,d}
\end{equation}
and
\begin{equation}
	D_{B,nd}=D_{A,nd}
\end{equation}

\subsubsection{Design C}
Because the total length of the diagonal and non-diagonal elements are the same as for Design A, we obtain the same diameters for the elements, namely,
\begin{equation}
D_{A,nd}=2D_{A,d}
\end{equation}
and
\begin{equation}
D_{C,nd}=D_{A,nd}
\end{equation}

\subsubsection{Design D}
For this design, because it does not contain any diagonal elements, we must allocate the total volume of the structure fully on the non-diagonal elements. Thus we can formulate the total volume of Design A as
\begin{equation}
	V_{A,nd}+V_{A,d}=(D_{A,nd}+\sqrt{2}D_{A,d})8Lt
\end{equation}
Using the relationship between the non-diagonal thickness and diagonal thickness, we obtain
\begin{equation}
V_{A,nd}+V_{A,d}=\left(1+{1\over\sqrt{2}}\right)8LtD_{A,nd}
\end{equation}

We can build a relationship between Design A and Design D through the total volume, such that the total volume of Design A is the same as the total volume of Design D
\begin{equation}
\left(1+{1\over\sqrt{2}}\right)8LtD_{A,nd}=8LtD_{D,nd}
\end{equation}
which yields a relationship between the thicknesses of the non diagonal elements as
\begin{equation}
	D_{D,nd}=\left(1+{1\over\sqrt{2}}\right)D_{A,nd}
\end{equation}

\subsection{Cylindrical Cross-section}

\subsubsection{Design A}
Design A is inspired by the sponge structure and consists of a square grid  reinforced by a double diagonal system (see \cref{TwoDiag}). If we assume that the horizontal and vertical struts have length $L$ and circular cross section  with diameters $D_{A,nd}$, their volume and projected-area in the unit cell are given by 
\begin{equation}\label{V1}
V_{A,nd}=8 L \left(\pi\frac{D^2_{A,nd}}{4}\right)=2 L \pi D^2_{A,nd}
\end{equation}
and
\begin{equation}\label{A1}
A_{A,nd}=8 L D_{A,nd},
\end{equation}
respectively. Note that in this study we use 
\begin{equation}
\frac{D_{A,nd}}{L}=0.1,
\end{equation}
since this is the aspect ratio measured for the sponges (see \cref{sec:params}).

Moreover, as for the case of the sponge, the diagonal elements are assumed to form an octagonal opening on every other cell. As such, the diagonals intersect the horizontal and vertical struts at a distance $\Delta L=L/(\sqrt{2}+2)$ from the nodes and their  volume and projected area in the unit cell are
\begin{equation}\label{V2}
V_{A,d}=8 \sqrt{2} L\left(\pi\frac{D^2_{A,d}}{4}\right)=2\sqrt{2}L \pi D^2_{A,d},
\end{equation}
and
\begin{equation}\label{A2}
A_{A,d}=8\sqrt{2} L D_{A,d},
\end{equation}
respectively. Since the projected area ratio of the non-diagonal to diagonal elements in the sponge has been measured to be 
\begin{equation}
\frac{A_{A,nd}}{A_{A,d}}=1.4,
\end{equation}
by substituting \cref{A1} and \cref{A2} into the equation above we find that for Design A
\begin{equation} \label{DA}
D_{A,nd}=1.4\sqrt{2}D_{A,d}\approx 2 D_{A,d}.
\end{equation}
Substitution of \cref{DA} into \cref{V1} and \cref{V2} yields 
\begin{equation}
\frac{V_{A,nd}}{V_{A,d}}=\frac{2 L \pi D^2_{A,nd}}{2\sqrt{2}L \pi D^2_{A,d}}=2\sqrt{2}
\end{equation}
and
\begin{equation}\label{VT}
V_{A,T}=V_{A,nd}+V_{A,d}=2\pi L (D_{A,nd}^2+\sqrt{2} D_{A,d}^2)=2\pi L D_{A,nd}^2 \left(1+\frac{1}{2\sqrt{2}}\right),
\end{equation}
where $V_{A,T}$ indicates the total volume of the unit cell for Design A. 
% such that \cref{VT} can be further simplified to
% \begin{equation}
% {V}_{A,T}=2\pi L D_{A,nd}^2 \left(1+\frac{1}{2\sqrt{2}}\right)\approx1.3288[\text{mm}^3].
% \end{equation}

Finally, it is important to note that in this study we use Design A as our base model, and thus constrain the total volume of all the other unit cell designs to be equal to that of Design A, namely,
\begin{equation}\label{con1}
V_{\alpha,d}+V_{\alpha,nd}={V}_{A,T}=2\pi L D_{A,nd}^2 \left(1+\frac{1}{2\sqrt{2}}\right),
\end{equation}
with $\alpha=$ B, C and D.
Finally,  for Designs B and C, which comprise diagonal elements, we also  constrain the volume ratio of the non-diagonal to diagonal elements to be the same as in Design A
\begin{equation}\label{con2}
\frac{V_{\alpha,nd}}{V_{\alpha,d}}=\frac{V_{A,nd}}{V_{A,d}}=2\sqrt{2},
\end{equation}
with $\alpha=$ B and C.

\figll{TwoDiag}{0.45}{{\bf Unit cell for Design A.} This design is inspired by the sponge structure and consists of a square grid  reinforced by a double diagonal system. The horizontal and vertical struts have length $L$ and circular cross section with diameter $D_{A,nd}$ and, as with the sponge, we assume $D_{A,nd}/L=0.1$. The diagonal elements have a circular cross section  with diameter $D_{A,d}=2 D_{A,nd}$.}

\subsubsection{Design B}
Design B is similar to the sponge design (Design A) and is likewise characterized by an alternation of open and closed cells (\cref{OneDiag}). However, instead of having two diagonals offset from the nodes, here we only have one diagonal passing through the nodes and crossing every other cell. It follows that the non-diagonal and diagonal volumes are given by
\begin{equation}
V_{B,nd}=V_{A,nd}=2\pi L D_{B,nd}^2
\end{equation}
and
\begin{equation}
V_{B,d}=2\sqrt{8} L \left(\pi \frac{{D}_{B,d}^2}{4}\right),
\end{equation}
respectively.
Using the constraints provided by \cref{con1} and \cref{con2}, as well as the above volumes, we  obtain 
\begin{equation}
{{D}_{B,nd}}={{D}_{A,nd}}
\end{equation}
and
\begin{equation}
\frac{{D}_{B,d}}{{D}_{B,nd}}=\frac{1}{\sqrt{2}}.
\end{equation}

 \figll{OneDiag}{0.45}{{\bf Unit cell for Design B.} This design is still characterized by an alternation of open and closed cells. However, instead of having two  diagonals offset from the nodes, here we only have one diagonal passing through the nodes and crossing every other cell. The horizontal and vertical struts have length $L$ and a circular cross section with diameter $D_{B,nd}$. The diagonal elements have a circular cross section  with diameter $D_{B,d}={D_{B,nd}}/{\sqrt{2}}$.}

\subsubsection{Design C}
Design C is inspired by the town lattice truss design introduced by architect Ithiel Town in 1820 \citep{waddell1916} and consists of every cell being reinforced by diagonal trusses passing through the nodes (see \cref{FullDiag}). For this design, the non-diagonal and diagonal volumes of the unit cell are given by:
\begin{equation}
V_{C,nd}=V_{A,nd}=2 L \pi D^2_{A,nd}
\end{equation}
and
\begin{equation}
V_{C,d}=V_{A,d}=2\sqrt{2}L \pi D^2_{A,d},
\end{equation}
respectively.
Using the constraints provided by \cref{con1} and \cref{con2} we  obtain 
\begin{equation}
{{D}_{C,nd}}={{D}_{A,nd}}
\end{equation}
and
\begin{equation}
\frac{{D}_{C,d}}{{D}_{C,nd}}=\frac{1}{2}.
\end{equation}


\figll{FullDiag}{0.45}{{\bf Unit cell for Design C.} This design consists of a square grid with all cells being reinforced by diagonal trusses passing through the nodes.  The horizontal and vertical struts have length $L$ and a circular cross section with diameter $D_{C,nd}$. The diagonal elements have a circular cross section  with diameter $D_{C,d}={D_{C,nd}}/{2}$.}

\subsubsection{Design D} 
Design D comprises only the square grid without diagonal reinforcement (\cref{NoDiag}) and is well known to be unstable and very limited in resisting shear forces \citep{gibson1999,deshpande2001}. As such, for this design we allocate the total material volume to the non-diagonal elements. Since
\begin{equation}
V_{D,T}=V_{D,nd}=V_{A,nd}=2\pi L D_{D,nd}^2,
\end{equation}
using the constraint provided by \cref{con1} we obtain
\begin{equation}
{D_{D,nd}}={D_{A,nd}}\sqrt{1+\frac{\sqrt{2}}{4}}. 
\end{equation}

\figll{NoDiag}{0.45}{{\bf Unit cell for Design D.} This design consists of a square grid without diagonal reinforcement.  The horizontal and vertical struts have length $L$ and a circular cross section with diameter $D_{D,nd}$.}

%\subsection{Analytical Derivation Of Stiffness}
%\subsection{Analytical Derivation of Critical Buckling Strain}
%\subsection{Derivation for }

% Bibliography
\nocite{aizenberg2005}
\nocite{deshpande2001}
\nocite{miserez2008}
\nocite{weaver2010}

\bibliography{refs}
\bibliographystyle{apalike}
% \bibliographystyle{plainnat}

\end{document}